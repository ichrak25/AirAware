\documentclass[12pt,a4paper]{article}

% Packages
\usepackage[utf8]{inputenc}
\usepackage[english]{babel}
\usepackage{graphicx}
\usepackage{hyperref}
\usepackage{listings}
\usepackage{xcolor}
\usepackage{geometry}
\usepackage{fancyhdr}
\usepackage{tocloft}
\usepackage{titlesec}
\usepackage{float}

% Page geometry
\geometry{
a4paper,
left=2.5cm,
right=2.5cm,
top=3cm,
bottom=3cm
}

% Listings configuration for code
\lstset{
basicstyle=\ttfamily\small,
breaklines=true,
frame=single,
language=Python,
showstringspaces=false,
commentstyle=\color{gray},
keywordstyle=\color{blue},
stringstyle=\color{red},
numbers=left,
numberstyle=\tiny\color{gray},
stepnumber=1,
numbersep=5pt,
backgroundcolor=\color{white},
tabsize=4,
captionpos=b
}

% Header and footer
\pagestyle{fancy}
\fancyhf{}
\rhead{\thepage}
\lhead{Scope Statement}
\lhead{AIR-AWARE: Outdoor Air Quality Intelligent Surveillance System}
\renewcommand{\headrulewidth}{0.4pt}

\begin{document}

% -------------------- Page de Garde ----------------------
\newcommand\titleofdoc{\bfseries Rapport PM }
\begin{titlepage}
\centering

\vspace*{2cm} % * force l'espacement même dans titlepage
\includegraphics[width=0.4\textwidth]{supcom.png}%

\vspace{2cm}

\vspace{1cm}
\rule{\textwidth}{0.4pt}
\vspace{0.3cm}

\begin{center}
{\Huge \textbf{\scshape Scope Statement}}
\\

{\Huge \textbf{\scshape AIR-AWARE: Outdoor Air Quality Intelligent Surveillance System}}
\end{center}

\vspace{0.3cm}
\rule{\textwidth}{0.4pt}

\vspace{1.5cm}

\begin{center}
\textbf{\textit{Realised by :}}\\[0.2cm]
{Fatma Abid \& Ichrak Aouadni}\\[0.2cm]
{Ahmed Guermazi \& Adem Bakey}\\[0.2cm]
\vspace{1cm}

\textbf{\textit{Faculty Advisor :}}\\[0.2cm]
Mohamed Becha Kaaniche
\end{center}

\vfill

{\large\textbf{2025/2026} }
\end{titlepage}

\tableofcontents
\newpage

\section{Concept}
\subsection{Project Context}
Outdoor air quality (OAQ) has become a growing public health concern as individuals are exposed to air pollutants in urban, industrial, and traffic-prone environments. Numerous studies conducted by the World Health Organization (WHO) and environmental agencies have highlighted that exposure to pollutants such as particulate matter (PM$_{2.5}$/PM$_{10}$), carbon dioxide (CO$_{2}$), volatile organic compounds (VOCs), and carbon monoxide (CO) can have serious effects on health, including respiratory, cardiovascular, and allergic diseases.  

In addition to health implications, poor outdoor air quality negatively impacts environmental comfort, urban planning, and public awareness. Conventional air quality monitoring stations are typically expensive, sparse, and unable to provide localized, real-time data.  

In this context, the AIR-AWARE project proposes the design and implementation of an Intelligent Outdoor Air Quality Surveillance System that combines IoT technologies, artificial intelligence, and cloud-based analytics to monitor, analyze, and predict variations in outdoor air quality. This system empowers authorities, researchers, and citizens with real-time information and actionable recommendations for a healthier outdoor environment.

\subsection{Problem Statement}
Despite increasing awareness of environmental health, outdoor air pollution remains a critical concern in urban and industrial areas. Many public spaces, streets, parks, and residential zones are exposed to high levels of PM$_{2.5}$/PM$_{10}$, CO$_{2}$, CO, and VOCs, which degrade air quality and increase the risk of respiratory and cardiovascular diseases.  

Existing outdoor monitoring solutions are often limited to a few fixed stations, are expensive, and provide low spatial resolution. This prevents proactive management of air quality and hinders timely public alerts. There is a critical need for a comprehensive, intelligent, and cost-effective system capable of continuously monitoring outdoor air quality, detecting anomalies, and providing predictive insights.  

The AIR-AWARE system is designed to fill this gap by combining IoT-based sensing, weatherproof hardware, cloud analytics, and AI-powered predictive models, with user-friendly interfaces for both authorities and the public.

\subsection{Objectives}

\subsubsection{Principal Objective}
The main objective of the AIR-AWARE project is to design and develop an intelligent ecosystem for real-time monitoring, analysis, and improvement of outdoor air quality. This ecosystem integrates IoT sensors, cloud-based data management, and AI models to ensure continuous surveillance, predictive analytics, and actionable recommendations aimed at maintaining healthy and safe outdoor environments.

\subsubsection{Specific Objectives}
To achieve this overarching goal, the project defines the following specific objectives:

\begin{enumerate}
  \item \textbf{Continuous Measurement:} Implement a network of weatherproof IoT sensors capable of accurately measuring key outdoor air parameters such as PM$_{2.5}$/PM$_{10}$, CO$_{2}$, CO, VOCs, temperature, and humidity.
  
  \item \textbf{Real-Time Analysis and Alerts:} Develop a data processing pipeline that performs real-time validation, anomaly detection, and triggers instant alerts when pollutant levels exceed predefined thresholds.
  
  \item \textbf{Predictive Modeling:} Employ machine learning algorithms (e.g., ARIMA, LSTM, Random Forest) to forecast short-term air quality variations and identify long-term pollution patterns.
  
  \item \textbf{User-Centric Visualization:} Design a Progressive Web Application (PWA) providing intuitive dashboards, historical data visualization, and personalized recommendations based on user preferences and environmental context.
  
  \item \textbf{Data Management and Scalability:} Implement a secure cloud backend using a scalable NoSQL database (MongoDB) to handle large volumes of sensor data, ensuring reliability, performance, and fault tolerance.
  
  \item \textbf{Location Tracking via GPS/LBS:} Integrate GPS/Location-Based Services (LBS) to enable mapping of outdoor air quality data, geolocation of devices, and contextual analysis across multiple monitored zones.
  
  \item \textbf{Autonomous Operation:} Introduce solar-assisted battery systems to maintain continuous operation in outdoor environments without relying solely on grid power.
  
  \item \textbf{Sustainability and Accessibility:} Deliver a cost-effective and energy-efficient solution accessible for urban, industrial, and residential outdoor areas.
\end{enumerate}

% -- Le reste du document reste inchangé, simplement remplacer "indoor" par "outdoor" partout, et mentionner weatherproof et autonomie solaire/batterie.

\section{Scope and Target Markets}

The AIR-AWARE project falls within the domain of intelligent indoor and outdoor air quality monitoring systems, combining Internet of Things (IoT) technologies, Artificial Intelligence (AI), and Location-Based Services (LBS) to provide a comprehensive, connected, and predictive solution.

The project scope covers the entire cycle of data collection, processing, visualization, and analysis of environmental parameters obtained from sensors deployed in various indoor and outdoor environments.

The system targets residential, professional, educational, industrial, and public outdoor spaces where air quality directly impacts health, comfort, and productivity. Thanks to its modular and scalable architecture, AIR-AWARE can be deployed in small domestic areas, large multi-site infrastructures, and outdoor urban or rural environments.

\subsection{Primary Markets}

The primary markets include sectors where air quality monitoring is critical for public health, comfort, and operational efficiency:

\subsubsection{Residential Sector}
Homes, apartments, and residential buildings requiring continuous air quality monitoring. Families, elderly individuals, or people with respiratory conditions seeking personalized, real-time insights through a mobile interface. Outdoor areas such as private gardens or courtyards can also be monitored.

\subsubsection{Corporate and Office Environments}
Offices, coworking spaces, and administrative buildings where air quality directly affects employee productivity and well-being. Outdoor areas like terraces, parking lots, or campus zones can also benefit from monitoring.

\subsubsection{Educational Institutions}
Classrooms, laboratories, and lecture halls requiring optimal air conditions. Outdoor playgrounds, sports fields, and campus areas can also be monitored for environmental safety.

\subsubsection{Healthcare and Medical Facilities}
Hospitals, clinics, laboratories, and retirement homes where clean air is essential. Outdoor waiting areas, gardens, and hospital entrances can also be monitored for pollution levels.

\subsection{Secondary Markets}

The secondary markets represent areas where the system can be adapted or extended to deliver additional value beyond its core applications:

\subsubsection{Industrial and Logistics Sector}
Factories, warehouses, and production zones requiring precise monitoring of indoor and outdoor gas emissions, dust, and chemical pollutants. Integration with energy management or industrial safety systems ensures compliance and operational efficiency.

\subsubsection{Hospitality and Commercial Sector}
Hotels, restaurants, shopping malls, and fitness centers where indoor and outdoor air quality directly influences customer satisfaction and experience.

\subsubsection{Public Institutions and Municipalities}
City halls, public schools, libraries, cultural centers, parks, streets, and outdoor gathering spaces aiming to deploy large-scale environmental monitoring networks for community well-being.

\subsubsection{Research and Innovation Sector}
Universities and research laboratories focusing on environmental modeling, urban health, and smart cities, using AIR-AWARE as a research and experimentation platform for both indoor and outdoor monitoring.

\section{Functional Needs}

The AIR-AWARE system is designed to ensure real-time monitoring, analysis, and control of indoor and outdoor air quality through an intelligent, connected network of IoT devices and data-driven software components.

The functional needs are organized into three main categories: data acquisition, data processing and analysis, and user interaction and alerting.

\subsection{Data Acquisition and Sensing Layer}

This layer is responsible for collecting environmental data through multiple sensors integrated into the AIR-AWARE device.

\textbf{Key functionalities:}
\begin{itemize}
  \item Continuous measurement of air quality parameters, including:
  \begin{itemize}
    \item Temperature
    \item Humidity
    \item Carbon Dioxide (CO$_{2}$) concentration
    \item Carbon Monoxide (CO) concentration
    \item Volatile Organic Compounds (VOCs)
    \item Particulate Matter (PM$_{2.5}$ / PM$_{10}$)
    \item Outdoor pollutants (NO$_{2}$, O$_{3}$, SO$_{2}$)
  \end{itemize}
  \item Automatic calibration and synchronization of sensors to ensure data accuracy and reliability.
  \item Data collection at configurable time intervals, depending on the use case (e.g., every 30 seconds or every 5 minutes).
  \item Integration with Location-Based Services (LBS) and GPS for dynamic tracking of sensor positions in large buildings, multi-zone installations, and outdoor urban or rural areas.
  \item Protective outdoor housing (IP65+) for weather resistance and durability.
\end{itemize}

\subsection{Data Processing and Intelligence Layer}

This layer performs data aggregation, cleaning, and analysis using intelligent algorithms to interpret environmental conditions.

\textbf{Key functionalities:}
\begin{itemize}
  \item Local preprocessing on the edge device (microcontroller or Raspberry Pi) to reduce latency and bandwidth use.
  \item Centralized data storage in the cloud for long-term historical analysis.
  \item Application of AI and Machine Learning models to:
  \begin{itemize}
    \item Detect anomalies and identify pollution peaks in indoor and outdoor environments.
    \item Predict air quality degradation trends considering weather, traffic, and seasonal effects.
    \item Recommend corrective actions such as ventilation, purification, or public warnings.
  \end{itemize}
  \item Real-time alert generation based on threshold violations or predictive risk patterns.
  \item Automatic synchronization between edge and cloud components to maintain data consistency.
\end{itemize}

\subsection{User Interaction and Visualization Layer}

This layer ensures that users can easily access, understand, and act upon air quality data through an intuitive interface.

\textbf{Key functionalities:}
\begin{itemize}
  \item A PWA displaying:
  \begin{itemize}
    \item Real-time sensor readings.
    \item Historical graphs and trend visualizations.
    \item Air quality index (AQI) indicators with color-coded severity levels.
    \item Interactive maps showing indoor and outdoor air quality variations across rooms, buildings, streets, and parks.
  \end{itemize}
  \item Push notifications and SMS/email alerts when critical thresholds are exceeded.
  \item Location-based visualization, showing air quality variations across different zones.
  \item Personalized user profiles allowing threshold customization and alert preferences.
  \item Administrative access for system management, device registration, and user control.
\end{itemize}

\subsection{Communication and Integration Layer}

\textbf{Key functionalities:}
\begin{itemize}
  \item Support for Wi-Fi, Bluetooth Low Energy (BLE), LoRaWAN, LTE-M/NB-IoT for long-range outdoor communication.
  \item Secure MQTT or HTTP-based APIs for transmitting data between devices and the cloud.
  \item Compatibility with third-party platforms (e.g., Google Cloud IoT, AWS IoT Core, municipal smart city platforms) for scalability and interoperability.
\end{itemize}

\subsection{Maintenance and System Reliability}

\textbf{Key functionalities:}
\begin{itemize}
  \item Self-diagnostic procedures to detect sensor malfunction or data loss.
  \item Over-the-air (OTA) firmware updates for easy maintenance.
  \item Encrypted data transmission and user authentication to guarantee security and privacy.
  \item Redundancy mechanisms and backup storage to prevent data loss.
  \item Regular maintenance alerts for outdoor sensors (dust, debris, extreme weather exposure).
\end{itemize}

\section{Non-Functional Requirements}

\subsection{Performance and Efficiency}
\begin{itemize}
  \item Real-time data collection and processing with latency < 2 seconds between sensor measurement and cloud synchronization.
  \item Sensors optimized for low-power consumption, with long-term deployment (>6 months battery life for low-power devices, solar support for outdoor nodes).
  \item Data processing and AI modules must handle large volumes of indoor and outdoor sensor data without performance degradation.
  \item Dashboard must refresh instantly for smooth user experience.
\end{itemize}

\subsection{Reliability and Availability}
\begin{itemize}
  \item Availability rate ≥ 99\% to ensure continuous monitoring.
  \item Offline data storage for network outages and automatic resynchronization.
  \item Outdoor hardware designed to withstand weather conditions, with periodic self-testing.
  \item Redundant cloud backups to prevent data loss.
\end{itemize}

\subsection{Security and Privacy}
\begin{itemize}
  \item End-to-end encryption (TLS 1.3) for all communications.
  \item Secure storage of user credentials and sensitive data (hashed and encrypted).
  \item Admin access restricted to authorized users.
  \item Compliance with GDPR and environmental data privacy regulations.
  \item Outdoor location data anonymized to prevent misuse.
\end{itemize}

\subsection{Scalability and Extensibility}
\begin{itemize}
  \item Modular architecture for integration of new sensors, outdoor analytics modules, and communication protocols.
  \item Horizontal scalability for city-wide sensor networks.
  \item Loosely coupled APIs and microservices for integration with external platforms.
\end{itemize}

\subsection{Maintainability and Upgradability}
\begin{itemize}
  \item OTA updates for sensors and gateways.
  \item Clean architecture principles for maintainable code.
  \item Comprehensive logging and diagnostics.
  \item Detailed documentation for indoor and outdoor deployments.
\end{itemize}

\subsection{Usability and Accessibility}
\begin{itemize}
  \item Intuitive, responsive, and accessible interface (WCAG 2.1).
  \item Easy configuration of thresholds, alerts, and visualizations.
  \item Consistent mobile and desktop experiences.
  \item Color coding, icons, graphs for clear interpretation.
  \item Multi-language support.
\end{itemize}

\subsection{Interoperability}
\begin{itemize}
  \item Support MQTT, CoAP, HTTP, LoRaWAN, LTE-M/NB-IoT.
  \item Interoperable with third-party services and smart city platforms.
  \item Data export in standard formats (JSON, CSV) for analysis or reporting.
\end{itemize}

\section{Machine Learning Integration}

AIR-AWARE leverages ML to enhance indoor and outdoor air quality monitoring, provide predictive insights, and trigger proactive alerts.

\subsection{Model Development}
\begin{itemize}
  \item Input variables: CO$_{2}$, VOCs, PM$_{2.5}$/PM$_{10}$, NO$_2$, O$_3$, SO$_2$, temperature, humidity, wind, traffic, GPS location.
  \item Predictive output: anomaly detection score or AQI for indoor and outdoor zones.
  \item Algorithm choice: ensemble methods, regression models, or neural networks optimized for real-time inference.
  \item Training data: historical indoor/outdoor sensor readings, weather, traffic, labeled events.
\end{itemize}

\subsection{MLOps Implementation}
\begin{itemize}
  \item Automated retraining and deployment.
  \item Continuous monitoring of model performance (accuracy, recall, false positives).
  \item Real-time alert triggers for potential hazards.
  \item Data versioning and management.
  \item Security and compliance for user and environmental data.
  \item Performance optimization with hyperparameter tuning.
  \item Feedback loop incorporating user and environmental feedback.
\end{itemize}

\section{Equipment}

To ensure precise air quality monitoring and real-time alerting, the following components will be used in AIR-AWARE:

\subsection{Raspberry Pi 4 Model B}
A versatile single-board computer with multiple GPIO pins, supporting Python, Java, and C/C++. Built-in Wi-Fi and Bluetooth make it ideal for IoT projects. It provides sufficient processing power for real-time sensor data collection, processing, and cloud communication.

\subsection{BME680 Environmental Sensor}
Measures temperature, humidity, pressure, and indoor air quality (IAQ) by detecting volatile organic compounds (VOCs). Communicates via I2C or SPI and is suitable for monitoring environmental conditions that may affect air quality.

\subsection{SGP30 Gas Sensor}
Detects indoor air quality, including CO$_2$ and total VOCs. Provides additional data to identify potential air hazards, enhancing anomaly detection in combination with other sensors.

\subsection{Particulate Matter Sensor (PMS5003 / SDS011)}
Measures PM$_{2.5}$ and PM$_{10}$ particles in the air. Critical for monitoring fine particulate pollution and assessing health risks.

\subsection{CO$_2$ Sensor (MH-Z19 or equivalent)}
Monitors carbon dioxide concentrations in indoor environments, providing early warning for poor air quality conditions.

\subsection{Cables and Connectors}
High-quality wiring to ensure stable communication between sensors and the Raspberry Pi, reducing noise and data loss.

\subsection{Power Supply and Housing}
Reliable power adapters and protective enclosures to maintain system stability and durability, ensuring continuous operation in different environments.

These components collectively form the core hardware infrastructure of AIR-AWARE, enabling precise environmental monitoring, seamless IoT integration, and accurate real-time data analysis.

\section{Technologies Choice}

\subsection{Back-end}
\begin{itemize}
  \item \textbf{MongoDB:} A NoSQL document-oriented database for storing sensor readings, user profiles, and historical air quality data. Offers flexibility, scalability, and easy integration with IoT devices.
  \item \textbf{MQTT:} Lightweight publish-subscribe protocol for transmitting real-time sensor data from AIR-AWARE devices to the cloud server efficiently and reliably.
\end{itemize}

\subsection{Middleware}
\begin{itemize}
  \item \textbf{JAX-RS:} Java API for RESTful web services, enabling secure and scalable APIs to fetch or push air quality data to applications.
  \item \textbf{WildFly:} Java EE application server for hosting middleware services, providing high reliability, load balancing, and enterprise-grade performance.
  \item \textbf{Mosquitto:} MQTT broker facilitating communication between IoT sensors and back-end systems.
\end{itemize}

\subsection{Front-End}
\begin{itemize}
  \item \textbf{Progressive Web App (PWA):} A web-based interface for users to monitor real-time air quality, receive alerts, and access historical data from any device without installing a native app.
  \item \textbf{Interactive Dashboards:} Visualizes sensor readings, air quality indices, and trends over time for better decision-making.
\end{itemize}

\subsection{IoT Integration}
\begin{itemize}
  \item \textbf{Flogo:} Lightweight, event-driven framework to orchestrate data flows, preprocess sensor data, and automate alerts in IoT environments.
  \item \textbf{ESP32/Raspberry Pi Integration:} Connects sensors to the IoT network, enabling real-time data collection, processing, and cloud synchronization.
\end{itemize}

\subsection{MLOps Integration}
\begin{itemize}
  \item \textbf{CI/CD Pipeline (Jenkins):} Automates the deployment and updates of machine learning models for predicting air quality hazards.
  \item \textbf{Testing Frameworks (PyTest, Selenium):} Ensures reliable model performance and accurate alert generation.
  \item \textbf{Monitoring (Prometheus \& Grafana):} Tracks system health, model predictions, and sensor anomalies.
  \item \textbf{Containerization (Docker):} Facilitates consistent deployment across multiple devices and servers.
  \item \textbf{Stream Processing (Apache Kafka):} Handles real-time sensor data streams for timely alerts and analytics.
  \item \textbf{Security (HashiCorp Vault):} Protects user data and sensor information with encryption and access control.
\end{itemize}
\section{Architecture}
\begin{figure}[h]
    \centering
    \includegraphics[width=15cm]{architecture.png}
    \caption{Architecture}
    \label{fig:Radar_steps}
\end{figure}
The diagram above describes the main architecture of the Aire-Aware monitoring system which
is mainly composed by Backend, Middleware and Frontend.

\section{Timeline and Tasks (8 Weeks)}
\begin{figure}[H]
    \centering
    \includegraphics[width=0.7\textwidth]{planning.png}
    \caption{Timeline and Tasks}
    \label{fig:roc_curve}
\end{figure}

\begin{tabular}{|c|p{14cm}|}
\hline
\textbf{Week} & \textbf{Tasks} \\
\hline
1 & Project kickoff, requirement analysis, and defining user stories. Research existing air quality monitoring systems and finalize hardware and software components. \\
\hline
2 & Sensor setup and initial IoT network configuration (ESP32/Raspberry Pi). Outdoor sensor tests and housing selection. Backend setup (MongoDB \& MQTT broker). \\
\hline
3 & Develop middleware services using JAX-RS and integrate MQTT communication. Begin basic front-end PWA setup with indoor and outdoor mapping. \\
\hline
4 & Implement data collection from sensors, real-time monitoring, and storage. Develop initial alert notification system. Outdoor deployment tests. \\
\hline
5 & Machine learning model development for air quality prediction. Begin MLOps setup including CI/CD pipeline, testing, and monitoring tools. \\
\hline
6 & Integrate ML model with backend and front-end dashboards. Implement Location-Based Services (LBS) and GPS for outdoor alerts. \\
\hline
7 & System testing: sensor accuracy, real-time alerts, dashboard functionality, and ML predictions. Outdoor environmental testing. Debug and optimize system performance. \\
\hline
8 & Finalize documentation, create prototype demonstration, and prepare presentation/video of the AIR-AWARE system. Deliverables submission. \\
\hline
\end{tabular}

\section{Methodology}

For the development of the AIR-AWARE system, we will employ Extreme Programming (XP), an Agile software development framework renowned for its adaptability, high-quality output, and focus on customer satisfaction. XP is particularly suitable for short-duration projects where requirements may evolve during the development process.

\subsection{Principles}

The methodology is guided by the following principles:

\begin{itemize}
  \item \textbf{High responsiveness to changing needs:} Quickly adapt to any modifications in user requirements or environmental considerations.
  \item \textbf{Team collaboration:} Pair programming and frequent communication to ensure a shared understanding of project goals.
  \item \textbf{Quality-focused development:} Emphasis on clean, maintainable code and early, frequent testing.
  \item \textbf{Continuous feedback:} Incorporate feedback from stakeholders, sensor testing, and ML predictions to improve the system.
\end{itemize}

\subsection{Process}

The XP framework normally involves 5 phases or stages of the development process that
iterate continuously:

\begin{itemize}
  \item \textbf{Planning:} Define user stories, functional and non-functional requirements, and desired outcomes. Create an 8-week release plan breaking down the system into manageable iterations.
  \item \textbf{Design:} Develop a simple, logical system architecture for IoT integration, backend, frontend, and ML components. Focus on clarity and maintainability.
  \item \textbf{Coding:} Implement the system using best practices, continuous integration, pair programming, and collective code ownership. Develop backend services, IoT communication, ML model integration, and front-end interfaces.
  \item \textbf{Testing:} Conduct ongoing testing, including unit tests for individual components, integration tests for IoT-backend-frontend communication, and acceptance tests with users. Validate ML model predictions and alert accuracy.
  \item \textbf{Listening:} Maintain continuous communication with stakeholders to incorporate feedback, clarify requirements, and ensure alignment with project objectives.
\end{itemize}
\section{Business Study}

The Business Study section provides an overall view of the AIR-AWARE system’s business model and marketing strategy.

\subsection{Business Model Canvas (BMC)}

The Business Model Canvas visually represents the main aspects of AIR-AWARE, including value proposition, customer segments, channels, cost structure, and revenue streams.
\begin{figure}[H]
    \centering
    \includegraphics[width=1\textwidth]{bmc.png} % Replace with your image filename
    \caption{Business Model Canvas}
    \label{fig:roc_curve}
\end{figure}
\subsection{Marketing Policy}

The marketing strategy for AIR-AWARE focuses on positioning it as a reliable, high-value air quality monitoring solution:

\subsubsection{Product}
\begin{itemize}
  \item A smart air monitoring system equipped with high-precision air quality sensors for real-time detection of pollutants and hazardous gases.
  \item A Progressive Web Application (PWA) for immediate alerts, historical data visualization, and location-based monitoring.
  \item Continuous enhancement of ML algorithms for accurate predictive alerts and reduced false positives.
\end{itemize}

\subsubsection{Price}
\begin{itemize}
  \item Competitive pricing with flexible plans for individual, commercial, and institutional customers.
  \item Discounts for bulk installations, long-term service agreements, and educational institutions.
\end{itemize}

\subsubsection{Promotion}
\begin{itemize}
  \item Multi-channel promotion through social media, digital marketing campaigns, and participation in health and environmental technology trade shows.
  \item Free trials, webinars, and educational content to raise awareness of air quality monitoring benefits.
\end{itemize}

\subsubsection{Place}
\begin{itemize}
  \item Accessible via web, iOS, and Android apps, providing users with convenient access to air quality data and alerts from anywhere.
\end{itemize}
\section{Limitations}
\begin{itemize}
  \item Internet connectivity dependence.
  \item Sensor accuracy affected by extreme outdoor conditions.
  \item Device placement critical for accuracy.
  \item High data volume management.
  \item Environmental interference (dust, humidity, temperature, wind).
  \item User response dependency.
  \item Outdoor risk of vandalism or theft.
\end{itemize}

\section{Deliverables}

The AIR-AWARE project will produce the following key outputs:

\begin{itemize}
  \item \textbf{Conceptual Document:} A detailed document outlining the system specifications, features, and services provided, including sensor network architecture, IoT integration, and user interface design.
  \item \textbf{Source Code:} Complete source code for the backend, middleware, frontend, and IoT integration components, hosted on GitHub for transparency and collaboration.
  \item \textbf{Prototype/Simulation:} A functional prototype or simulation demonstrating real-time air quality monitoring, notifications, and location tracking via LBS.
  \item \textbf{Demonstration Video:} A video (mp4 format) showcasing the system’s functionalities, including real-time alerts, dashboard interface, and sensor data visualization.
  \item \textbf{User Documentation:} Guides for installation, setup, and operation of the AIR-AWARE system for end-users, including troubleshooting tips.
  \item \textbf{Technical Documentation:} Comprehensive documentation for developers, detailing architecture, APIs, and system workflows for maintenance and future upgrades.
\end{itemize}

\end{document}
